\documentclass[11pt]{article}
\textheight 30cm
\textwidth 18.5cm
\oddsidemargin -1cm
\evensidemargin -1cm
\topmargin -3cm

\setlength{\parindent}{0pt}
\setlength{\parskip}{2ex plus 0.5ex minus 0.2ex}
\usepackage{cite}
%To make the bibliography look right
\makeatletter \renewcommand\@biblabel[1]{} \makeatother
\renewcommand{\citeleft}{(}
\renewcommand{\citeright}{)}


\usepackage[english]{babel}
\usepackage[T1]{fontenc}
\usepackage[latin1]{inputenc}
\usepackage{mathptmx}
\usepackage{charter}
\usepackage{color}
\pagestyle{empty}
\begin{document}
\begin{center}
{\huge Trackit R package}
\end{center}
\Large
\thispagestyle{empty}
Trackit is an add-on package for the statistical environment {\bf R}
to efficiently estimate movement parameters and predict ``the most
probable track'' directly from the raw light measurements obtained
from archival tagged individuals. Its basic use is to fit and plot the
model described in \citen{Nielsen2007}.

The model is unique in estimating the most probable track without
making any light-level threshold assumptions, or constraining the
movement of the tag between dawn and dusk. The new model generates two
estimates of geographic positions per day (at dawn and dusk). The
covariance structure of the model is designed to handle high
correlations between light measurements, such as might be caused by
local weather conditions. The yearly pattern in latitude precision is
estimated by propagating the data uncertainties through the
geolocation process. The model has been applied to simulated data,
mooring studies, and real deployments on swimming and diving fish.

An attempt has been made to make it as simple as possible to apply 
this model. For instance, to fit a basic track all that is necessary 
is to bring data into the right format: 
\vspace{-0.6cm}
{\color{blue}
\begin{small}
\begin{verbatim}
    year  month   day     hour    min     sec     depth   light   temp
    2003  6       29      2       23      0       0       158     0
    2003  6       29      2       35      0       0       156     0
       .  .        .      .        .      .       .         .     .
       .  .        .      .        .      .       .         .     .
       .  .        .      .        .      .       .         .     .
\end{verbatim}
\end{small}}
and type the following commands into R: 
\vspace{-.3cm}
{\color{red}
\begin{small}
\begin{verbatim}
    library(trackit)
    track<-read.table('filename.txt')
    prepared.track<-prepit(track,
                           fix.first=c(242.12,32.87,2003,6,28,0,0,0),
                           fix.last=c(240.42,32.62,2003,12,27,0,0,0))
    fit<-trackit(prepared.track)
    plot(fit)
\end{verbatim}
\end{small}}
First line loads the package. Second line reads the data. Third 
line (here spanning three lines) prepares the track and supplies 
information about known release and recapture positions. Fourth line 
does the model optimization. Finally the fifth line plots the most 
probable track.     

The importance of open collaboration for the future development and
application of electronic tags cannot be overemphasized. The model is
developed because a broad network of collaborating tag users and
manufacturers generously shared their time, data, and ideas.
\vspace{-1cm}
\begin{thebibliography}{}
\vspace{-0.75cm}
\bibitem[Nielsen~and~Sibert~2007]{Nielsen2007}
Nielsen,~A. and Sibert,~J.~R.~(2007).
{State--space model for light--based tracking of marine animals}.
{\it Can. J. Fish. Aquat. Sci.}~{\bf 64(8)}:1055-1068. 
\end{thebibliography}
%%%%%%%%%%%%%%%%%%%%%%%%%%%%%%%%%%%%%%%%%%%%%%%%%%%%%%%%%%%%%%%%%%%%%%
\newpage 
\begin{center}
{\huge Sea Surface Temperature in Trackit}
\end{center}
\Large
\thispagestyle{empty}
A fairly recent addition to the trackit package is the ability to 
include sea surface temperatures derived from temperature measurements 
in the tag to assist the geolocation. In periods where light--based 
geolocation is less precise (around equinoxes), this can be a great 
improvement. 

In order to use sea surface temperatures a second data frame must be 
prepared containing the columns: {\color{blue}\verb#year#},
{\color{blue}\verb#month#}, {\color{blue}\verb#day#}, 
{\color{blue}\verb#hour#}, {\color{blue}\verb#min#}, 
{\color{blue}\verb#sec#}, and {\color{blue}\verb#sst#}. Assuming this 
data is in the file {\color{blue}\verb#sstfile.txt#} a syntax example 
for fitting the model is:   
{\color{red}
\begin{small}
\begin{verbatim}
    library(trackit)
    track<-read.table('filename.txt')
    sstdata<-read.table('sstfile.txt')
    path<-get.sst.from.server(track, 220, 260, 10, 40)
    prepared.track<-prepit(track, sst=sstdata,
                           fix.first=c(242.12,32.87,2003,6,28,0,0,0),
                           fix.last=c(240.42,32.62,2003,12,27,0,0,0))
    fit<-trackit(prepared.track)
    plot(fit)
\end{verbatim}
\end{small}}
Most lines are similar to fitting a track with light alone. Third 
line reads in the SST data extracted from the tag. Fourth line 
downloads satellite SST images for the a user defined area, spanning 
the duration of the tag's deployment (several sources are available).  

The model is very flexible in that it does not require SST data to be 
available on all days where light data is available (or the other way 
around), so even if SST measurements from the tag are only available 
on a few days throughout the track - it can still be useful for the 
geolocation process.  

The option to include SST in the trackit package is developed in 
collaboration with Chi Lam from University of Southern California.
\end{document}

\newpage
\begin{center}
{\huge A simulated example}
\end{center}
\Large
\thispagestyle{empty}
In addition to estimating parameters and reconstructing most probable 
tracks from real data the packages also offers the ability of simulating 
data from the model. This serves several purposes. First it can be used 
to test the model implementation, and to verify that the parameters in 
the model are in fact identifiable. Second it is a useful (and 
inexpensive) tool for experimenting with tag settings and experimental 
designs. Examples of relevant questions could be: How much would the 
precision in geolocation improve by using SST in my study area? How does 
different sampling rate settings influence the precision in geolocation? 
What precision can be expected near equinoxes in my area? My marine 
creatures tend loose their tags fast, with such short deployments how 
precise can I expect inference about the model parameters to be?       

The function used to simulate data from the model is called 
{\color{blue}\verb#light.simulator#}. This function has a lot of default 
settings,


set.seed(1234567)
library(trackit)
x<-date.mdy(mdy.date(2,1,2003)+1:150)
sstdates<-as.data.frame(cbind(x$year, x$month, x$day, 0,0,0))
sim<-light.simulator(sstdates=sstdates)
fit<-trackit(sim)


\end{document}
